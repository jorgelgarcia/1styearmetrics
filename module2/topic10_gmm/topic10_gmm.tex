%Input preamble
%Style
\documentclass[12pt]{article}
\usepackage[top=1in, bottom=1in, left=1in, right=1in]{geometry}
\parindent 22pt
\usepackage{fancyhdr}

%Packages
\usepackage{adjustbox}
\usepackage{amsmath}
\usepackage{amsfonts}
\usepackage{amssymb}
\usepackage{bm}
\usepackage[table]{xcolor}
\usepackage{tabu}
\usepackage{color,soul}
\usepackage{makecell}
\usepackage{longtable}
\usepackage{multirow}
\usepackage[normalem]{ulem}
\usepackage{etoolbox}
\usepackage{graphicx}
\usepackage{tabularx}
\usepackage{ragged2e}
\usepackage{booktabs}
\usepackage{caption}
\usepackage{fixltx2e}
\usepackage[para, flushleft]{threeparttablex}
\usepackage[capposition=top,objectset=centering]{floatrow}
\usepackage{subcaption}
\usepackage{pdfpages}
\usepackage{pdflscape}
\usepackage{natbib}
\usepackage{bibunits}
\definecolor{maroon}{HTML}{990012}
\usepackage[bottom]{footmisc}
\usepackage[colorlinks=true,linkcolor=maroon,citecolor=maroon,urlcolor=maroon,anchorcolor=maroon]{hyperref}
\usepackage{marvosym}
\usepackage{makeidx}
\usepackage{tikz}
\usetikzlibrary{shapes}
\usepackage{setspace}
\usepackage{enumerate}
\usepackage{rotating}
\usepackage{tocloft}
\usepackage{epstopdf}
\usepackage[titletoc]{appendix}
\usepackage{framed}
\usepackage{comment}
\usepackage{xr}
\usepackage{titlesec}
\usepackage{footnote}
\usepackage{longtable}
\newlength{\tablewidth}
\setlength{\tablewidth}{9.3in}
\setcounter{secnumdepth}{4}
\usepackage{textgreek}

\titleformat{\paragraph}
{\normalfont\normalsize\bfseries}{\theparagraph}{1em}{}
\titlespacing*{\paragraph}
{0pt}{3.25ex plus 1ex minus .2ex}{1.5ex plus .2ex}
\makeatletter
\pretocmd\start@align
{%
  \let\everycr\CT@everycr
  \CT@start
}{}{}
\apptocmd{\endalign}{\CT@end}{}{}
\makeatother
%Watermark
\usepackage[printwatermark]{xwatermark}
\usepackage{lipsum}
\definecolor{lightgray}{RGB}{220,220,220}
%\newwatermark[allpages,color=lightgray,angle=45,scale=3,xpos=0,ypos=0]{Preliminary Draft}

%Further subsection level
\usepackage{titlesec}
\setcounter{secnumdepth}{4}
\titleformat{\paragraph}
{\normalfont\normalsize\bfseries}{\theparagraph}{1em}{}
\titlespacing*{\paragraph}
{0pt}{3.25ex plus 1ex minus .2ex}{1.5ex plus .2ex}

\setcounter{secnumdepth}{5}
\titleformat{\subparagraph}
{\normalfont\normalsize\bfseries}{\thesubparagraph}{1em}{}
\titlespacing*{\subparagraph}
{0pt}{3.25ex plus 1ex minus .2ex}{1.5ex plus .2ex}

%Functions
\DeclareMathOperator{\cov}{Cov}
\DeclareMathOperator{\corr}{Corr}
\DeclareMathOperator{\var}{Var}
\DeclareMathOperator{\plim}{plim}
\DeclareMathOperator*{\argmin}{arg\,min}
\DeclareMathOperator*{\argmax}{arg\,max}
\DeclareMathOperator{\supp}{supp}

%Math Environments
\newtheorem{theorem}{Theorem}
\newtheorem{claim}{Claim}
\newtheorem{condition}{Condition}
\renewcommand\thecondition{C--\arabic{condition}}
\newtheorem{algorithm}{Algorithm}
\newtheorem{assumption}{Assumption}
\newtheorem{remark}{Remark}
\renewcommand\theremark{R--\arabic{remark}}
\newtheorem{definition}[theorem]{Definition}
\newtheorem{hypothesis}[theorem]{Hypothesis}
\newtheorem{property}[theorem]{Property}
\newtheorem{example}[theorem]{Example}
\newtheorem{result}[theorem]{Result}
\newenvironment{proof}{\textbf{Proof:}}{$\bullet$}

%Commands
\newcommand\independent{\protect\mathpalette{\protect\independenT}{\perp}}
\def\independenT#1#2{\mathrel{\rlap{$#1#2$}\mkern2mu{#1#2}}}
\newcommand{\overbar}[1]{\mkern 1.5mu\overline{\mkern-1.5mu#1\mkern-1.5mu}\mkern 1.5mu}
\newcommand{\equald}{\ensuremath{\overset{d}{=}}}
\captionsetup[table]{skip=10pt}
%\makeindex

\setlength\parindent{20pt}
\setlength{\parskip}{0pt}

\newcolumntype{L}[1]{>{\raggedright\let\newline\\\arraybackslash\hspace{0pt}}m{#1}}
\newcolumntype{C}[1]{>{\centering\let\newline\\\arraybackslash\hspace{0pt}}m{#1}}
\newcolumntype{R}[1]{>{\raggedleft\let\newline\\\arraybackslash\hspace{0pt}}m{#1}}



%Logo
%\AddToShipoutPictureBG{%
%  \AtPageUpperLeft{\raisebox{-\height}{\includegraphics[width=1.5cm]{uchicago.png}}}
%}

\newcolumntype{L}[1]{>{\raggedright\let\newline\\\arraybackslash\hspace{0pt}}m{#1}}
\newcolumntype{C}[1]{>{\centering\let\newline\\\arraybackslash\hspace{0pt}}m{#1}}
\newcolumntype{R}[1]{>{\raggedleft\let\newline\\\arraybackslash\hspace{0pt}}m{#1}}

\newcommand{\mr}{\multirow}
\newcommand{\mc}{\multicolumn}

%\newcommand{\comment}[1]{}

\let\counterwithout\relax
\let\counterwithin\relax
\definecolor{maroon}{HTML}{4B0082}

\begin{document}

\noindent \textbf{Generalized Method of Moments.}\\
\noindent Jorge Luis García \\
\noindent e-mail: jlgarcia@tamu.edu\\

\noindent Throughout Econ 900-02 and Econ 900-03, two estimators have been considered: OLS, IV, and ML. This estimators have a common feature: They exploit moment conditions motivated either by an algorithm (e.g.,\ first-order conditions of minimization or maximization problems) or an assumption (e.g.,\ an orthogonality condition).\\

\noindent \textbf{OLS as a Method of Moments Estimator.} The first-order conditions of OLS lead to impose the empirical counterpart of the moment condition implied by the \textbf{Exogeneity} assumption. Thus, the OLS estimator is identical the the method-of-moments estimator based on the moment: 
\begin{align}
	\mathbb{E} \left[ \bm{x}_i' e_i \right] = \bm{0} 
\end{align}
\noindent with sample analog 
\begin{align}
	\frac{1}{N} \cdot \sum \limits _{i \in \mathcal{I}} \bm{x}_i' \hat{e}_i & =  \frac{1}{N} \cdot \sum \limits _{i \in \mathcal{I}} \bm{x}_i' \left( y - \bm{x}_i \cdot \bm{\hat{\beta}} \right)  
	\nonumber \\ 
	& = \bm{0}
\end{align}
\noindent which yields $\bm{\hat{\beta}}^{\text{OLS}}$. In Econ 900-02, an analogous instrumental-variable estimator was formulated to exploit an orthogonality condition over an an instrumental variable $z_{i}$.\\

\noindent \textbf{ML as a Method of Moments Estimator.} ML is also a a method-of-moment estimator. To see this, recall that ML estimators ($\bm{\hat{\theta}}$) solve 
\begin{align}
	\frac{1}{N} \frac{ \partial \ln L} { \partial \bm{\hat{\theta}} } & = \frac{1}{N}   \sum \limits _{i \in \mathcal{I}} \left[ \frac{ \partial \ln f \left( y_i \mid \bm{x}_i, \bm{\hat{\theta}} \right) } { \partial \bm{\hat{\theta}} } \right] \nonumber \\ 
		& = \bm{0}
\end{align}
\noindent with population analog 
\begin{align}
	\mathbb{E} \left[ \frac{ \partial \ln f \left( y_i \mid \bm{x}_i, \bm{\theta} \right) } { \partial \bm{\theta} } \right] = \bm{0}
\end{align}

\noindent \textbf{Generalizing the Method of Moments (GMM).} Suppose that a model is characterizes by the K-dimensional parameter vector $\bm{\theta}$. Theory provides $L \geq K$ moment conditions: 
\begin{align}
	\mathbb{E} \left[ m_\ell \left( y_i, \bm{z}_i, \theta \right) \right] = 0 \text{ for } \ell = 1, \ldots L. 
\end{align}
\noindent with corresponding sample analog
\begin{align}
	\bar{m}_\ell \left( \bm{\theta} \right) & = \frac{1}{N} \sum \limits _{i \in \mathcal{I}} m_\ell \left( y_i, \bm{z}_i, \theta \right) \nonumber \\ & = 0 \text{ for } \ell = 1, \ldots L. 
\end{align}

\noindent If the $L$ equations are linearly independent, then there are multiple solution for $\bm{\theta}$. A possibility to obtain a unique solution is to minimize a quadratic criterion. For example, $\tilde{q} = \bar{\bm{m}}  \left( \bm{\theta} \right)' \bar{\bm{m}}  \left( \bm{\theta} \right)$, where $\bar{\bm{m}}  \left( \bm{\theta} \right)$ stacks the $L$ moment conditions. More generally, a weighted quadratic criterion is $q = \bar{\bm{m}}  \left( \bm{\theta} \right)' W_n \bar{\bm{m}}  \left( \bm{\theta} \right)$. In an analogy to the comparison between OLS and GLS, both the weighted and unweighted criterion yield consistent estimates. Generally, only the weighted criterion yields efficient estimates.\\

\noindent In the criterion $\tilde{q}$, the moments are assumed to be homoskedastic. The weighted criterion $q$ allows the moments to correlate. Either criteria yield consistent estimates, so usually $W_n$ is chosen so that it minimizes the variance of $\bm{\hat{\theta}}$.  This is the generalizes method of moments with variance $W_n = V^{-1}$ with $V = \frac{1}{N} \left[ \bm{\Gamma}' \bm{\Phi} \bm{\Gamma} \right]$, $\bm{\Gamma}^j = \plim \left( \frac{ \partial \bar{m}_j} { \partial \bm{\theta} } \right)$, and $\bm{\Phi} = \var \left( \bm{\bar{m}} \right)$. The variance of $\bm{\hat{\theta}}$ could be obtained using the Delta method or the bootstraps.
  
\end{document}
