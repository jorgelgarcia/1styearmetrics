%Input preamble
%Style
\documentclass[12pt]{article}
\usepackage[top=1in, bottom=1in, left=1in, right=1in]{geometry}
\parindent 22pt
\usepackage{fancyhdr}

%Packages
\usepackage{adjustbox}
\usepackage{amsmath}
\usepackage{amsfonts}
\usepackage{amssymb}
\usepackage{bm}
\usepackage[table]{xcolor}
\usepackage{tabu}
\usepackage{color,soul}
\usepackage{makecell}
\usepackage{longtable}
\usepackage{multirow}
\usepackage[normalem]{ulem}
\usepackage{etoolbox}
\usepackage{graphicx}
\usepackage{tabularx}
\usepackage{ragged2e}
\usepackage{booktabs}
\usepackage{caption}
\usepackage{fixltx2e}
\usepackage[para, flushleft]{threeparttablex}
\usepackage[capposition=top,objectset=centering]{floatrow}
\usepackage{subcaption}
\usepackage{pdfpages}
\usepackage{pdflscape}
\usepackage{natbib}
\usepackage{bibunits}
\definecolor{maroon}{HTML}{990012}
\usepackage[bottom]{footmisc}
\usepackage[colorlinks=true,linkcolor=maroon,citecolor=maroon,urlcolor=maroon,anchorcolor=maroon]{hyperref}
\usepackage{marvosym}
\usepackage{makeidx}
\usepackage{tikz}
\usetikzlibrary{shapes}
\usepackage{setspace}
\usepackage{enumerate}
\usepackage{rotating}
\usepackage{tocloft}
\usepackage{epstopdf}
\usepackage[titletoc]{appendix}
\usepackage{framed}
\usepackage{comment}
\usepackage{xr}
\usepackage{titlesec}
\usepackage{footnote}
\usepackage{longtable}
\newlength{\tablewidth}
\setlength{\tablewidth}{9.3in}
\setcounter{secnumdepth}{4}
\usepackage{textgreek}

\titleformat{\paragraph}
{\normalfont\normalsize\bfseries}{\theparagraph}{1em}{}
\titlespacing*{\paragraph}
{0pt}{3.25ex plus 1ex minus .2ex}{1.5ex plus .2ex}
\makeatletter
\pretocmd\start@align
{%
  \let\everycr\CT@everycr
  \CT@start
}{}{}
\apptocmd{\endalign}{\CT@end}{}{}
\makeatother
%Watermark
\usepackage[printwatermark]{xwatermark}
\usepackage{lipsum}
\definecolor{lightgray}{RGB}{220,220,220}
%\newwatermark[allpages,color=lightgray,angle=45,scale=3,xpos=0,ypos=0]{Preliminary Draft}

%Further subsection level
\usepackage{titlesec}
\setcounter{secnumdepth}{4}
\titleformat{\paragraph}
{\normalfont\normalsize\bfseries}{\theparagraph}{1em}{}
\titlespacing*{\paragraph}
{0pt}{3.25ex plus 1ex minus .2ex}{1.5ex plus .2ex}

\setcounter{secnumdepth}{5}
\titleformat{\subparagraph}
{\normalfont\normalsize\bfseries}{\thesubparagraph}{1em}{}
\titlespacing*{\subparagraph}
{0pt}{3.25ex plus 1ex minus .2ex}{1.5ex plus .2ex}

%Functions
\DeclareMathOperator{\cov}{Cov}
\DeclareMathOperator{\corr}{Corr}
\DeclareMathOperator{\var}{Var}
\DeclareMathOperator{\plim}{plim}
\DeclareMathOperator*{\argmin}{arg\,min}
\DeclareMathOperator*{\argmax}{arg\,max}
\DeclareMathOperator{\supp}{supp}

%Math Environments
\newtheorem{theorem}{Theorem}
\newtheorem{claim}{Claim}
\newtheorem{condition}{Condition}
\renewcommand\thecondition{C--\arabic{condition}}
\newtheorem{algorithm}{Algorithm}
\newtheorem{assumption}{Assumption}
\newtheorem{remark}{Remark}
\renewcommand\theremark{R--\arabic{remark}}
\newtheorem{definition}[theorem]{Definition}
\newtheorem{hypothesis}[theorem]{Hypothesis}
\newtheorem{property}[theorem]{Property}
\newtheorem{example}[theorem]{Example}
\newtheorem{result}[theorem]{Result}
\newenvironment{proof}{\textbf{Proof:}}{$\bullet$}

%Commands
\newcommand\independent{\protect\mathpalette{\protect\independenT}{\perp}}
\def\independenT#1#2{\mathrel{\rlap{$#1#2$}\mkern2mu{#1#2}}}
\newcommand{\overbar}[1]{\mkern 1.5mu\overline{\mkern-1.5mu#1\mkern-1.5mu}\mkern 1.5mu}
\newcommand{\equald}{\ensuremath{\overset{d}{=}}}
\captionsetup[table]{skip=10pt}
%\makeindex

\setlength\parindent{20pt}
\setlength{\parskip}{0pt}

\newcolumntype{L}[1]{>{\raggedright\let\newline\\\arraybackslash\hspace{0pt}}m{#1}}
\newcolumntype{C}[1]{>{\centering\let\newline\\\arraybackslash\hspace{0pt}}m{#1}}
\newcolumntype{R}[1]{>{\raggedleft\let\newline\\\arraybackslash\hspace{0pt}}m{#1}}



%Logo
%\AddToShipoutPictureBG{%
%  \AtPageUpperLeft{\raisebox{-\height}{\includegraphics[width=1.5cm]{uchicago.png}}}
%}

\newcolumntype{L}[1]{>{\raggedright\let\newline\\\arraybackslash\hspace{0pt}}m{#1}}
\newcolumntype{C}[1]{>{\centering\let\newline\\\arraybackslash\hspace{0pt}}m{#1}}
\newcolumntype{R}[1]{>{\raggedleft\let\newline\\\arraybackslash\hspace{0pt}}m{#1}}

\newcommand{\mr}{\multirow}
\newcommand{\mc}{\multicolumn}

%\newcommand{\comment}[1]{}

\let\counterwithout\relax
\let\counterwithin\relax
\definecolor{maroon}{HTML}{4B0082}

\begin{document}

\onehalfspacing
\noindent \textbf{Selection on Observables.}\\
\noindent Jorge Luis García \\
\noindent e-mail: jlgarcia@tamu.edu\\

\noindent \textbf{Context and Notation.} Let $D_i$ indicate treatment status for individual $i$. The \textit{discrete} support of this variable represents treatment categories. In general, this support is a subset of the non-negative integer numbers (categories), $\mathcal{Z}_{+}$. Let the support be denoted by $\mathcal{D}$, where $\mathcal{D} \subseteq \mathcal{Z}_{+}$.\\

\noindent A canonical example is when $\mathcal{D}:= \{ 0,1 \}$, where $0$ indicates no college (no treatment) and $1$ indicates college (treatment). Economists analyze the ``college-going'' decision through the random variable $D_i$. The random variable takes either of the values in $\mathcal{D}$. Let $d$ denote the realization of $D_i$. In the college example, either $d = 0$ or $d = 1$.\\

\noindent \textbf{Selection into Treatment.} Let $y_{i}^d$ denote an economic outcome of interest for individual $i$ when treatment status is $d$. For example, labor income when not going to college, $y_{i}^0$, or labor income when going to college, $y_{i}^1$. Importantly, the economist only gets to observe data for an individual in either of the states but \textit{not in both}. That is, they only observe $y_{i}^0$ or $y_{i}^1$. This is the \textit{substance of the program evaluation problem}. If the economist could observe the individual in both states, then the evaluation would be a trivial exercise facilitated by the computable comparison of $y_{i}^1$ to $y_{i}^0$.\\

\noindent There is \textit{selection into treatment} when $\left( y^{d} | D_i = d \right) \neq \left( y^{d} | D_i = d' \right) $ for $d \neq d'$ with $d, d' \in \mathcal{D}$. Note that  $y^{d} | D_i = d$ is the distribution of the economic outcome of interest when treatment is fixed to $d$, conditional on $D_i = d$. That is the distribution of the outcome when fixing treatment status to $d$ \textit{fundamentally} differs when $D_i = d$ if compared to $D_i = d'$.\\ 

\noindent Fixing and conditioning are crucial to distinguish in this context. Fixing is the conceptual exercise of ``fixing an individual'' to a counterfactual outcome. Conditioning is the standard statistical exercise. Generally, fixing and conditioning are only the same under randomized assignment of treatment.\\

\noindent \textbf{Switching Regressions and Treatment Effects in the Two-category Case.} Let $y_{i}$ denote the \textit{observed} outcome for individual $i$. The switching regression model of \citet{quandt_new_1972} formulates  
\begin{equation} 
y_{i} =  \left( 1 - D_{i} \right) \cdot y_{i}^0 + D_{i} \cdot y_{i}^1,
\end{equation}
\noindent noting that $y_{i} =  y_{i}^0$ if $D_i = 0$ and $y_{i} =  y_{i}^1$ if $D_i = 1$. The \textit{observed} expected difference between treated and untreated individuals is
\begin{align}
\mathbb{E} \left[ y_{i} | D_{i} = 1 \right] - \mathbb{E} \left[ y_{i} | D_{i} = 0 \right] &= \mathbb{E} \left[ y_{i}^1 | D_{i} = 1 \right] - \mathbb{E} \left[ y_{i}^0 | D_{i} = 0 \right] \nonumber \\
&= \underbrace{\left[ \mathbb{E} \left[ y_{i}^1 | D_{i} = 1 \right] - \mathbb{E} \left[ y_{i}^0 | D_{i} = 1 \right] \right]}_{\text{Average Treatment on the Treated (ATT)}} \nonumber \\ 
&+  \underbrace{\left[ \mathbb{E} \left[ y_{i}^0 | D_{i} = 1 \right] - \mathbb{E} \left[ y_{i}^0 | D_{i} = 0 \right] \right]}_{\text{Selection Bias}} \nonumber \\
&=  \text{ATT} + \text{Selection Bias}
\end{align}

\noindent The \textit{observed} expected difference between the treated and the untreated does not identify a program evaluation parameter, unless there is no selection bias (e.g.,\ randomization).\\

\noindent \textbf{Selection on Observables.} A common solution to identify parameters of interest is the conditional independence assumption (CIA). The assumption is so strong that it allows economists to identify the Average Treatment Effect (ATE), the ATT, and the Average Treatment on the Treated (ATU).\\ 

\noindent The CIA states that $y^d \independent D_{i} | \bm{X}_{i}$, where $ \bm{X}_i $ is a vector of observed characteristics. Other ways to refer to the CIA are unconfoundedness or ignorability (of treatment assignment).\\

\noindent \textbf{Identification of the ATE.} Under the CIA, the ATE is identified. To see this, note that 
\begin{eqnarray}
\text{ATE} &:=& \mathbb{E} \left[ y_i^1 -y_i^0 \right] \nonumber \\
		&=&  \mathbb{E}_{ \bm{X}_{i}} \left[ \mathbb{E} \left[ y_i^1 |  \bm{X}_{i} \right] -  \mathbb{E} \left[ y_i^0 |  \bm{X}_{i} \right] \right] \noindent \\ 
		&=& \mathbb{E}_{ \bm{X}_{i}} \left[ \mathbb{E} \left[ y_i | D_i = 1,  \bm{X}_{i} \right] -  \mathbb{E} \left[ y_i | D_i = 0,  \bm{X}_{i} \right] \right]
\end{eqnarray}

\noindent The third line follows because, if the CIA holds, fixing and conditioning on $D_{i}$ become equivalent once $ \bm{X}_{i}$ is conditioned on. $\mathbb{E} \left[ y_i | D_i = 1,  \bm{X}_{i} \right]$ and $\mathbb{E} \left[ y_i | D_i = 0,  \bm{X}_{i} \right]$ are observed in the data.\\

\noindent \textbf{Identification of the ATT.} Under the CIA, the ATT is identified. To see this, note that 
 \begin{eqnarray}
\text{ATT} &:=& \mathbb{E} \left[ y_i^1 - y_i^0 | D_i = 1 \right] \nonumber \\
		&=&  \mathbb{E} \left[ y_i^1 | D_i = 1 \right] -  \mathbb{E} \left[ y_i^0 | D_i = 1 \right] \noindent \\
		&=& \mathbb{E} \left[ y_i | D_i = 1 \right] -  \mathbb{E}_{ \bm{X}_{i}} \left[ \mathbb{E} \left[ y_i | D_i = 0,  \bm{X}_{i} \right] | D_i = 1 \right],
\end{eqnarray}

\noindent where the third line follows from the CIA. $\mathbb{E} \left[ y_i | D_i = 1 \right]$ is observed in the data and estimation methods, which allows for recovering $\mathbb{E}_{ \bm{X}_{i}} \left[ \mathbb{E} \left[ y_i | D_i = 0,  \bm{X}_{i} \right] | D_i = 1 \right]$ (see the matching and IPW notes). Identification of the ATU is analogous.\\

\noindent \textbf{Practical Aspects.} 
	\begin{itemize}
		\item When does the CIA hold?
			\begin{itemize}
				\item This is impossible to verify.
				\item Knowing the selection process and conditioning on variables describing it is ideal. 
			\end{itemize}
		\item The conditioning variables should be predetermined (\textit{ex ante}). Conditioning should be on selection, not on treatment. See \citet{neal_role_1996} for more on this. 
		\item Selection on observables or the CIA is a difficult assumption to defend because the assumption boils down to the two following simultaneous statements: (i) individual outcomes differ because of $D_i$; and (ii) $D_{i}$ is as random as a balanced coin flip once treatment \textit{ex ante} characteristics, $\bm{X}_{i}$, are conditioned on. An example that makes clear how far the strength of this assumption could take a researcher is \citet{heckman_effect_2010}. Good luck defending that!
	\end{itemize}

\singlespacing
\bibliographystyle{chicago}
\bibliography{BIB}
\end{document}



