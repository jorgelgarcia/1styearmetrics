%Input preamble
%Style
\documentclass[12pt]{article}
\usepackage[top=1in, bottom=1in, left=1in, right=1in]{geometry}
\parindent 22pt
\usepackage{fancyhdr}

%Packages
\usepackage{adjustbox}
\usepackage{amsmath}
\usepackage{amsfonts}
\usepackage{amssymb}
\usepackage{bm}
\usepackage[table]{xcolor}
\usepackage{tabu}
\usepackage{color,soul}
\usepackage{makecell}
\usepackage{longtable}
\usepackage{multirow}
\usepackage[normalem]{ulem}
\usepackage{etoolbox}
\usepackage{graphicx}
\usepackage{tabularx}
\usepackage{ragged2e}
\usepackage{booktabs}
\usepackage{caption}
\usepackage{fixltx2e}
\usepackage[para, flushleft]{threeparttablex}
\usepackage[capposition=top,objectset=centering]{floatrow}
\usepackage{subcaption}
\usepackage{pdfpages}
\usepackage{pdflscape}
\usepackage{natbib}
\usepackage{bibunits}
\definecolor{maroon}{HTML}{990012}
\usepackage[bottom]{footmisc}
\usepackage[colorlinks=true,linkcolor=maroon,citecolor=maroon,urlcolor=maroon,anchorcolor=maroon]{hyperref}
\usepackage{marvosym}
\usepackage{makeidx}
\usepackage{tikz}
\usetikzlibrary{shapes}
\usepackage{setspace}
\usepackage{enumerate}
\usepackage{rotating}
\usepackage{tocloft}
\usepackage{epstopdf}
\usepackage[titletoc]{appendix}
\usepackage{framed}
\usepackage{comment}
\usepackage{xr}
\usepackage{titlesec}
\usepackage{footnote}
\usepackage{longtable}
\newlength{\tablewidth}
\setlength{\tablewidth}{9.3in}
\setcounter{secnumdepth}{4}
\usepackage{textgreek}

\titleformat{\paragraph}
{\normalfont\normalsize\bfseries}{\theparagraph}{1em}{}
\titlespacing*{\paragraph}
{0pt}{3.25ex plus 1ex minus .2ex}{1.5ex plus .2ex}
\makeatletter
\pretocmd\start@align
{%
  \let\everycr\CT@everycr
  \CT@start
}{}{}
\apptocmd{\endalign}{\CT@end}{}{}
\makeatother
%Watermark
\usepackage[printwatermark]{xwatermark}
\usepackage{lipsum}
\definecolor{lightgray}{RGB}{220,220,220}
%\newwatermark[allpages,color=lightgray,angle=45,scale=3,xpos=0,ypos=0]{Preliminary Draft}

%Further subsection level
\usepackage{titlesec}
\setcounter{secnumdepth}{4}
\titleformat{\paragraph}
{\normalfont\normalsize\bfseries}{\theparagraph}{1em}{}
\titlespacing*{\paragraph}
{0pt}{3.25ex plus 1ex minus .2ex}{1.5ex plus .2ex}

\setcounter{secnumdepth}{5}
\titleformat{\subparagraph}
{\normalfont\normalsize\bfseries}{\thesubparagraph}{1em}{}
\titlespacing*{\subparagraph}
{0pt}{3.25ex plus 1ex minus .2ex}{1.5ex plus .2ex}

%Functions
\DeclareMathOperator{\cov}{Cov}
\DeclareMathOperator{\corr}{Corr}
\DeclareMathOperator{\var}{Var}
\DeclareMathOperator{\plim}{plim}
\DeclareMathOperator*{\argmin}{arg\,min}
\DeclareMathOperator*{\argmax}{arg\,max}
\DeclareMathOperator{\supp}{supp}

%Math Environments
\newtheorem{theorem}{Theorem}
\newtheorem{claim}{Claim}
\newtheorem{condition}{Condition}
\renewcommand\thecondition{C--\arabic{condition}}
\newtheorem{algorithm}{Algorithm}
\newtheorem{assumption}{Assumption}
\newtheorem{remark}{Remark}
\renewcommand\theremark{R--\arabic{remark}}
\newtheorem{definition}[theorem]{Definition}
\newtheorem{hypothesis}[theorem]{Hypothesis}
\newtheorem{property}[theorem]{Property}
\newtheorem{example}[theorem]{Example}
\newtheorem{result}[theorem]{Result}
\newenvironment{proof}{\textbf{Proof:}}{$\bullet$}

%Commands
\newcommand\independent{\protect\mathpalette{\protect\independenT}{\perp}}
\def\independenT#1#2{\mathrel{\rlap{$#1#2$}\mkern2mu{#1#2}}}
\newcommand{\overbar}[1]{\mkern 1.5mu\overline{\mkern-1.5mu#1\mkern-1.5mu}\mkern 1.5mu}
\newcommand{\equald}{\ensuremath{\overset{d}{=}}}
\captionsetup[table]{skip=10pt}
%\makeindex

\setlength\parindent{20pt}
\setlength{\parskip}{0pt}

\newcolumntype{L}[1]{>{\raggedright\let\newline\\\arraybackslash\hspace{0pt}}m{#1}}
\newcolumntype{C}[1]{>{\centering\let\newline\\\arraybackslash\hspace{0pt}}m{#1}}
\newcolumntype{R}[1]{>{\raggedleft\let\newline\\\arraybackslash\hspace{0pt}}m{#1}}



%Logo
%\AddToShipoutPictureBG{%
%  \AtPageUpperLeft{\raisebox{-\height}{\includegraphics[width=1.5cm]{uchicago.png}}}
%}

\newcolumntype{L}[1]{>{\raggedright\let\newline\\\arraybackslash\hspace{0pt}}m{#1}}
\newcolumntype{C}[1]{>{\centering\let\newline\\\arraybackslash\hspace{0pt}}m{#1}}
\newcolumntype{R}[1]{>{\raggedleft\let\newline\\\arraybackslash\hspace{0pt}}m{#1}}

\newcommand{\mr}{\multirow}
\newcommand{\mc}{\multicolumn}

%\newcommand{\comment}[1]{}

\let\counterwithout\relax
\let\counterwithin\relax
\definecolor{maroon}{HTML}{4B0082}

\begin{document}

\noindent \textbf{Regression Discontinuity Design.}\\
\noindent Jorge Luis García \\
\noindent e-mail: jlgarcia@tamu.edu\\

\noindent \textbf{Notation.} Let $y_{i} = y_{i}^0 + D_{i} \cdot \left( y_{i}^1 - y_{i}^0 \right)$ be the outcome of interest for individual $i \in \mathcal{I}$, where $\mathcal{I}$ indexes individuals. $D_{i}$ is a binary treatment indicator. Regression discontinuity design is a method developed to identify and estimate causal effects in a framework where there is a discontinuous relationship between the variables $R_{i}$ and $D_{i}$ at a point $c$. Specifically $\Pr \left[ D_{i} = 1 | R_i = r \right]$ is discontinuous at $r = c$. $R_{i}$ is commonly referred to as the running or forcing variable. $c$ is the cutoff.\\

\noindent A researcher is interested in implementing this method because they believe that $D_{i}$ and $R_{i}$ are correlated with unobserved characteristics. However, they have a design where it is plausible to assume that $y_{i}^d$ for $d = 0,1$ varies continuously with $R_i$ at $R_i = c$. Thus, at $R_i = c$ the most variation in the potential outcomes is due to the discontinuity in $\Pr \left[ D_{i} = 1 | R_i = r \right]$.\\

\noindent \textbf{Immediate Limitation.} By construction, this empirical strategy only identifies causal effects for compliers with $R_{i} = c$. Additionally, different cutoffs identify different causal effects of interest (draw an analogy with linear instrumental-variable estimators).\\ 

\noindent \textbf{Identification in Sharp Regression Discontinuity Designs.} In sharp designs, $D_{i}$ changes deterministically from 0 to 1: $D_{i} := \bm{1} \left[ R_{i} \geq c \right]$. Therefore, $\left( y_{i0}, y_{i1} \right) \independent D_{i} | R_{i}$. This allows for (non-parametric) identification of $\mathbb{E} \left[ y_{i0} | R_{i} = r\right]$ for $R_{i} < c$ and $\mathbb{E} \left[ y_{i1} | R_{i} = r\right]$ for $R_{i} \geq c$. If $y_{i}^d$ for $d = 0,1$ varies continuously with $R_i$ at $R_i = c$, then $\mathbb{E} \left[  y_{i1} - y_{i1} | R_{i} = c \right]$ is (non-parametrically) identified. Graphically, 

\begin{figure}[H]
\centering
\caption{Identification in Regression Discontinuity Designs, Illustration}
\includegraphics[width=.75\textwidth]{rdd.jpg}
\end{figure}

\noindent Formally, let $D_{i} := \bm{1} \left[ R_{i} \geq c \right]$ and let $\mathbb{E} \left[ y_{id} | R_{i} = r\right]$ be continuous at $r = c$. Then, 
\begin{eqnarray} 
\lim_{r \downarrow c} \mathbb{E} \left[ y_{i} | R_{i} = r \right] &=& \lim_{r \downarrow c}  \mathbb{E} \left[ y_{i1} | R_{i} = r \right] \nonumber \\
&=& \mathbb{E} \left[ y_{i1} | R_{i} = c \right], 
\end{eqnarray}
\noindent where the first equality follows because the design is sharp and the second because of continuity. Analogously, $\lim_{r \uparrow c} \mathbb{E} \left[ y_{i} | R_{i} = r \right] =  \mathbb{E} \left[ y_{i0} | R_{i} = c \right]$. Thus, 
\begin{equation}
\mathbb{E} \left[ y_{i1} - y_{i0} | R_{i} = c \right] = \lim_{r \downarrow c} \mathbb{E} \left[ y_{i} | R_{i} = r \right] - \lim_{r \uparrow c} \mathbb{E} \left[ y_{i} | R_{i} = r \right].
\end{equation}

\noindent Sharp RDD is an ATE local at $c$.\\

\noindent \textbf{Identification in Fuzzy Regression Discontinuity Designs.} In this case, $\Pr \left[ D_{i} = 1 | R_i = r \right]$ is discontinuous at $r = c$, but $\Pr \left[ D_{i} = 1 | R_i = r \right] \neq \bm{1} \left[ R_i \geq c \right]$. Thus, $ z_{i}:= \bm{1} \left[ R_i \leq c \right]$ is a relevant but not an exact predictor of  of $D_{i}$. Fuzzy regression discontinuity is a limiting form of instrumental variables in which $\left( y_{i0} - y_{i1} \right) \independent z_{i} | R_{i}$.\\

\noindent Formally, let $z_{i}$ satisfy the linear instrumental-variable setting assumptions and assume that $\mathbb{E} \left[ y_{id} | R_i = r, \kappa = k \right]$ and $\Pr \left[ \kappa = k | R_i = r \right]$ are continuous at $r = c$, where $\kappa = \{$ \text{always takers}, \text{never takers}, \text{compliers} $\}$, then

\begin{eqnarray}
\mathbb{E} \left[ y_{i1} - y_{i0} | R_i = c, \text{compliers} \right] &=& 
\frac{  \lim_{r \downarrow c} \mathbb{E} \left[ y_{i} | z_{i} = 1, R_{i} = r  \right] - \lim_{r \uparrow c} \mathbb{E} \left[ y_{i} | z_{i} = 0, R_{i} = r  \right] } {  \lim_{r \downarrow c} \mathbb{E} \left[ D_{i} | z_{i} = 1, R_{i} = r  \right] - \lim_{r \uparrow c} \mathbb{E} \left[ D_{i} | z_{i} = 0, R_{i} = r  \right] } \nonumber \\
&=&\frac{  \lim_{r \downarrow c} \mathbb{E} \left[ y_{i} | R_{i} = r  \right] - \lim_{r \uparrow c} \mathbb{E} \left[ y_{i} | R_{i} = r  \right] } {  \lim_{r \downarrow c} \mathbb{E} \left[ D_{i} | R_{i} = r  \right] - \lim_{r \uparrow c} \mathbb{E} \left[ D_{i} | R_{i} = r  \right] }.
\end{eqnarray}

\noindent where the equality follows because, at the limit, $z_i$ becomes redundant. Fuzzy RDD, therefore, identifies a LATE that is local to compliers at the cutoff, which is not necessarily a parameter of interest. This is also true for the parameter that sharp RDD identifies. Therefore, the critiques to LATE discussed in the linear IV case apply here.\\

\noindent \textbf{Practice.} Non-parametric identification at the cutoff suggests using individuals at the cutoff for estimation. However, $y_{i0}$ is not observed if $R_{i} \geq c$ and there could be few observations for whom $R_{i} = c$ to estimate $y_{i1}$. Thus, researchers usually employ observations for whom $R < c $ and $R \geq c $ to form separate estimators of $\mathbb{E} \left[ y_{i} | R_{i} \right] $.\\

\noindent Verifications of the empirical design include inspections of $\mathbb{E} \left[ D_i | R_i = r \right]$, $\mathbb{E} \left[ y_i | R_i = r \right]$ and $\mathbb{E} \left[ z_i | R_i = r \right]$ for falsification covariates.\\ 

\noindent \textbf{Implementation Basics.} In the sharp case, a regression allows for estimation of $\mathbb{E} \left[ y_{i}^d | R_{i} = r \right]$: 
\begin{equation}
y_{i} = \alpha + \theta \cdot D_{i} + g \left( R_i \right) + \varepsilon_{i}, 
\end{equation}
\noindent for an unknown functions $g \left( \cdot \right)$. Polynomials or other semi-parametric functions of $R_{i} - c$ are usual proxies. Importantly:

\begin{itemize}
\item Can be estimated as locally in the interval $R_{i} \in \left[ h - c, h + c \right]$ and let $h$ be as small as the data permits. 
\item Other covariates could be added. 
\item Standard errors are straightforward to recover (or if not use bootstrap)
\item Extends to fuzzy designs with $z_{i}$ as an instrument for $D_{i}$ in 
\begin{equation}
y_{i} = \alpha + \theta \cdot D_{i} + \theta_l \cdot \bm{1} \left[ R_i < c \right]  + \theta_h \cdot \bm{1} \left[ R_i \geq c \right]  + \varepsilon_{i}.
\end{equation}
\end{itemize}

\end{document}
