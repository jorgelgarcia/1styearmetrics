%Input preamble
%Style
\documentclass[12pt]{article}
\usepackage[top=1in, bottom=1in, left=1in, right=1in]{geometry}
\parindent 22pt
\usepackage{fancyhdr}

%Packages
\usepackage{adjustbox}
\usepackage{amsmath}
\usepackage{amsfonts}
\usepackage{amssymb}
\usepackage{bm}
\usepackage[table]{xcolor}
\usepackage{tabu}
\usepackage{color,soul}
\usepackage{makecell}
\usepackage{longtable}
\usepackage{multirow}
\usepackage[normalem]{ulem}
\usepackage{etoolbox}
\usepackage{graphicx}
\usepackage{tabularx}
\usepackage{ragged2e}
\usepackage{booktabs}
\usepackage{caption}
\usepackage{fixltx2e}
\usepackage[para, flushleft]{threeparttablex}
\usepackage[capposition=top,objectset=centering]{floatrow}
\usepackage{subcaption}
\usepackage{pdfpages}
\usepackage{pdflscape}
\usepackage{natbib}
\usepackage{bibunits}
\definecolor{maroon}{HTML}{990012}
\usepackage[bottom]{footmisc}
\usepackage[colorlinks=true,linkcolor=maroon,citecolor=maroon,urlcolor=maroon,anchorcolor=maroon]{hyperref}
\usepackage{marvosym}
\usepackage{makeidx}
\usepackage{tikz}
\usetikzlibrary{shapes}
\usepackage{setspace}
\usepackage{enumerate}
\usepackage{rotating}
\usepackage{tocloft}
\usepackage{epstopdf}
\usepackage[titletoc]{appendix}
\usepackage{framed}
\usepackage{comment}
\usepackage{xr}
\usepackage{titlesec}
\usepackage{footnote}
\usepackage{longtable}
\newlength{\tablewidth}
\setlength{\tablewidth}{9.3in}
\setcounter{secnumdepth}{4}
\usepackage{textgreek}

\titleformat{\paragraph}
{\normalfont\normalsize\bfseries}{\theparagraph}{1em}{}
\titlespacing*{\paragraph}
{0pt}{3.25ex plus 1ex minus .2ex}{1.5ex plus .2ex}
\makeatletter
\pretocmd\start@align
{%
  \let\everycr\CT@everycr
  \CT@start
}{}{}
\apptocmd{\endalign}{\CT@end}{}{}
\makeatother
%Watermark
\usepackage[printwatermark]{xwatermark}
\usepackage{lipsum}
\definecolor{lightgray}{RGB}{220,220,220}
%\newwatermark[allpages,color=lightgray,angle=45,scale=3,xpos=0,ypos=0]{Preliminary Draft}

%Further subsection level
\usepackage{titlesec}
\setcounter{secnumdepth}{4}
\titleformat{\paragraph}
{\normalfont\normalsize\bfseries}{\theparagraph}{1em}{}
\titlespacing*{\paragraph}
{0pt}{3.25ex plus 1ex minus .2ex}{1.5ex plus .2ex}

\setcounter{secnumdepth}{5}
\titleformat{\subparagraph}
{\normalfont\normalsize\bfseries}{\thesubparagraph}{1em}{}
\titlespacing*{\subparagraph}
{0pt}{3.25ex plus 1ex minus .2ex}{1.5ex plus .2ex}

%Functions
\DeclareMathOperator{\cov}{Cov}
\DeclareMathOperator{\corr}{Corr}
\DeclareMathOperator{\var}{Var}
\DeclareMathOperator{\plim}{plim}
\DeclareMathOperator*{\argmin}{arg\,min}
\DeclareMathOperator*{\argmax}{arg\,max}
\DeclareMathOperator{\supp}{supp}

%Math Environments
\newtheorem{theorem}{Theorem}
\newtheorem{claim}{Claim}
\newtheorem{condition}{Condition}
\renewcommand\thecondition{C--\arabic{condition}}
\newtheorem{algorithm}{Algorithm}
\newtheorem{assumption}{Assumption}
\newtheorem{remark}{Remark}
\renewcommand\theremark{R--\arabic{remark}}
\newtheorem{definition}[theorem]{Definition}
\newtheorem{hypothesis}[theorem]{Hypothesis}
\newtheorem{property}[theorem]{Property}
\newtheorem{example}[theorem]{Example}
\newtheorem{result}[theorem]{Result}
\newenvironment{proof}{\textbf{Proof:}}{$\bullet$}

%Commands
\newcommand\independent{\protect\mathpalette{\protect\independenT}{\perp}}
\def\independenT#1#2{\mathrel{\rlap{$#1#2$}\mkern2mu{#1#2}}}
\newcommand{\overbar}[1]{\mkern 1.5mu\overline{\mkern-1.5mu#1\mkern-1.5mu}\mkern 1.5mu}
\newcommand{\equald}{\ensuremath{\overset{d}{=}}}
\captionsetup[table]{skip=10pt}
%\makeindex

\setlength\parindent{20pt}
\setlength{\parskip}{0pt}

\newcolumntype{L}[1]{>{\raggedright\let\newline\\\arraybackslash\hspace{0pt}}m{#1}}
\newcolumntype{C}[1]{>{\centering\let\newline\\\arraybackslash\hspace{0pt}}m{#1}}
\newcolumntype{R}[1]{>{\raggedleft\let\newline\\\arraybackslash\hspace{0pt}}m{#1}}



%Logo
%\AddToShipoutPictureBG{%
%  \AtPageUpperLeft{\raisebox{-\height}{\includegraphics[width=1.5cm]{uchicago.png}}}
%}

\newcolumntype{L}[1]{>{\raggedright\let\newline\\\arraybackslash\hspace{0pt}}m{#1}}
\newcolumntype{C}[1]{>{\centering\let\newline\\\arraybackslash\hspace{0pt}}m{#1}}
\newcolumntype{R}[1]{>{\raggedleft\let\newline\\\arraybackslash\hspace{0pt}}m{#1}}

\newcommand{\mr}{\multirow}
\newcommand{\mc}{\multicolumn}

%\newcommand{\comment}[1]{}

\let\counterwithout\relax
\let\counterwithin\relax
\definecolor{maroon}{HTML}{4B0082}

\begin{document}
%\onehalfspacing

\noindent \textbf{Models for Binary-Response Dependent Variable.}\\
\noindent Jorge Luis García \\
\noindent e-mail: jlgarcia@tamu.edu\\

\noindent Models where $y_i$ is binary are useful because many of the decisions that economists study are of this nature (e.g., giving birth, getting into college, joining the labor force, retiring). It is usual to model the decision $y_i$ as
\begin{align}
	\Pr \left( y_i = 1 \mid \bm{x} \right) & = F \left( \bm{x}_i \cdot \bm{\beta} \right) \nonumber \\ 
	\Pr \left( y_i = 0 \mid \bm{x} \right) & = 1 - F \left( \bm{x}_i \cdot \bm{\beta} \right), 
\end{align}

\noindent for a cumulative density function $F \left( \bm{x}_i \cdot \bm{\beta} \right)$ that depends of explanatory variables $\bm{x}$ and is parameterized by $\bm{\beta}$.\\

\noindent \textbf{Linear-Probability Model.} The first possibility is to let $F$ be the identity mapping so that $F \left( \bm{x}_i \cdot \bm{\beta} \right) = \bm{x} \cdot \bm{\beta}$. In that case, 
\begin{align}
	\mathbb{E} \left[ y_i \mid \bm{x} \right] & = \Pr \left( y_i = 1 \mid \bm{x} \right) \nonumber \\  
											  & =  \bm{x}_i \cdot \bm{\beta}. 
\end{align} 
\noindent The linear-regression model for individual $i \in \mathcal{I}$ is
\begin{align}
	y_i & = \mathbb{E} \left[ y_i \mid \bm{x} \right] - \left( y_i - \mathbb{E} \left[ y_i \mid \bm{x} \right] \right) \nonumber \\ 
		& = \mathbb{E} \left[ y_i \mid \bm{x} \right] + e_i \nonumber \\
		& = \bm{x}_i \cdot \bm{\beta} + e_i.
\end{align}

\noindent Modeling a binary choice with with a linear-regression model (linear-probability model) is the current, standard practice in applied economics. This modeling has two \textit{specification issues}:

\begin{enumerate}
	\item $e_i \mid \bm{x}$ is heteroskedastic. This follows because $\bm{x}_i \cdot \bm{\beta} + e_i$ is either $0$ or $1$. Therefore, $e_i$ is binary: It is $- \bm{x}_i \cdot \bm{\beta}$ when $y_i = 0$ and $1 - \bm{x}_i \cdot \bm{\beta}$ when $y_i = 1$. Therefore, $\var\left( e_i \mid \bm{x} \right)  = \mathbb{E} \left[ y_i \mid \bm{x} \right] \cdot \left( 1 - \mathbb{E} \left[ y_i \mid \bm{x} \right]  \right) = \bm{x}_i \cdot \bm{\beta} \cdot \left( 1 - \bm{x}_i \cdot \bm{\beta} \right)$. This problem is straightforward.
	\item $\Pr \left( y_i = 1 \mid \bm{x} \right)$ is out of bounds. In this model, $\Pr \left( y_i = 1 \mid \bm{x} \right) = \mathbb{E} \left[ y_i \mid \bm{x} \right] = \bm{x}_i \cdot \bm{\beta}$. In the absence of restrictions on $\bm{x}$, the prediction of the model could be either negative or greater than $1$, which is a problem because the model aims to predict a probability. This is an issue that could be solved with careful specification of $\bm{x}$ (e.g., if $\bm{x}$ categorizes observed characteristics, then the estimated $\bm{\beta}$ are the sample averages of $y_i$ for each category). 
\end{enumerate}

\noindent \textbf{Probit and Logit Models.} An alternative is to model $F$ in a way such that $\Pr \left( y_i = 1 \mid \bm{x} \right)$ is appropriately bounded. Two usual options are 
\begin{align}
	\textbf{Probit} &: \Pr \left( y_i = 1 \mid \bm{x} \right)  = \Phi \left( \bm{x}_i \cdot \bm{\beta} \right)  \nonumber \\ 
	\textbf{Logit}  &: \Pr \left( y_i = 1 \mid \bm{x} \right)  = \Lambda \left( \bm{x}_i \cdot \bm{\beta} \right) \nonumber,  \\
\end{align}
\noindent where $\Phi$ and $\Lambda$ are the standard normal and standard logistic  cumulative density functions.\\

\noindent In either case, the coefficient vector $\bm{\beta}$ is not a marginal effect as in the case of linear regression. For a continuous regressor variable $k$, 
	\begin{align}
	\textbf{Marginal Effect in the Probit} &: \frac{\partial \Pr \left( y_i = 1 \mid \bm{x} \right)} {\partial x_{ik} } = \frac{\partial  \Phi \left( \bm{x} \cdot \bm{\beta} \right) } {\partial x_{ik} } = \phi\left( \bm{x} \cdot \bm{\beta} \right) \cdot \beta_k   \nonumber \\ 
	\textbf{Marginal Effect in the Logit} &: \frac{\partial \Pr \left( y_i = 1 \mid \bm{x} \right)} {\partial x_{ik} } = \frac{\partial  \Lambda \left( \bm{x} \cdot \bm{\beta} \right) } {\partial x_{ik} } = \lambda\left( \bm{x} \cdot \bm{\beta} \right) \cdot \beta_k \nonumber.
\end{align}

\noindent \textbf{Random Utility Models.} A general modeling of a binary decision is to let $U^k = \bm{x}_i \cdot \bm{\beta}^k + e^k$ be the utility when $y_i = k$ for $k = 0, 1$. Individual $i \in \mathcal{I}$ decides according the following rule: 
\begin{align}
y_i = \left\{
        \begin{array}{ll}
            0 & \quad U_i^0 > U_i^1 \\
            1 & \quad U_i^1 > U_i^0.
        \end{array}
    \right. \nonumber
\end{align}
\noindent Then, 
\begin{align}
	 \Pr \left( y_i = 1 \mid \bm{x} \right) &= \Pr \left( U^1 > U^0 \mid \bm{x} \right) \nonumber \\ 
	 & = \Pr \left( \bm{x}_i \cdot \left( \bm{\beta}^1 - \bm{\beta}^0 \right) + \left( e_i^1 - e_i^0 \right) > 0 \mid \bm{x} \right) \nonumber \\
	 & =: \Pr \left( \bm{x}_i \cdot \bm{\beta} + e_i > 0 \mid \bm{x} \right) \\ \nonumber
	 & =  \Pr \left( \frac{e_i}{\sigma} > - \frac{\bm{x}_i \cdot \bm{\beta}}{\sigma} \mid \bm{x} \right), 
\end{align}
\noindent where $\sigma:= \sqrt{\var \left( e_i \mid \bm{x} \right)}$. The last equality shows that it is not possible to separately identify the elements in $\bm{\beta}$ from $\sigma$. Thus the usual assumption that $\sigma = 1$. This assumption directly enables linking random utility models to either the Probit or Logit models.\\

\noindent \textbf{Estimation in Binary-Response Models.} Estimation in the linear probability model proceeds as thoroughly discussed in Econ 900-02. For Probit and Logit models, maximum likelihood enables estimating $\bm{\beta}$. For $F = \Phi, \Lambda$, the likelihood contribution of $i \in \mathcal{I}$
 \begin{align}
 	l_i & :=  \Pr \left( y_i = 1 \mid \bm{x} \right) \nonumber \\ 
 	& = {\left[ 1 - F \left( \bm{x}_i \cdot \bm{\beta} \right) \right]}^{\bm{1} \left( y_i = 0 \right)} \cdot {\left[F \left( \bm{x}_i \cdot \bm{\beta} \right) \right]}^{\bm{1} \left( y_i = 1 \right)}. 
 \end{align}
 \noindent The sample log-likelihood is 
 \begin{align}
 \ln L = \sum \limits _{i \in \mathcal{I}} \left[ y_i \cdot \ln F \left( \bm{x}_i \cdot \bm{\beta} \right) \right]  + \left[ \left( 1 - y_i \right) \cdot \ln \left( 1 - F \left( \bm{x}_i \cdot \bm{\beta} \right) \right) \right] 
 \end{align}

\noindent \textbf{Variance.} From Econ 900-02, 
\begin{align}
	\bm{\hat{\beta}^{\text{ML}}} \sim \mathcal{N} \left( \beta, I \left( \bm{\hat{\beta}^{\text{ML}}}  \right) \right). 
\end{align}
\noindent The marginal effects are functions of $\bm{\beta}$ so it is possible to apply the Delta Method to obtain standard errors. Likelihood ratio and Wald tests can also be implemented in this context. 
\end{document}
