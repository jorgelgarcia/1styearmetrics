%Input preamble
%Style
\documentclass[12pt]{article}
\usepackage[top=1in, bottom=1in, left=1in, right=1in]{geometry}
\parindent 22pt
\usepackage{fancyhdr}

%Packages
\usepackage{adjustbox}
\usepackage{amsmath}
\usepackage{amsfonts}
\usepackage{amssymb}
\usepackage{bm}
\usepackage[table]{xcolor}
\usepackage{tabu}
\usepackage{color,soul}
\usepackage{makecell}
\usepackage{longtable}
\usepackage{multirow}
\usepackage[normalem]{ulem}
\usepackage{etoolbox}
\usepackage{graphicx}
\usepackage{tabularx}
\usepackage{ragged2e}
\usepackage{booktabs}
\usepackage{caption}
\usepackage{fixltx2e}
\usepackage[para, flushleft]{threeparttablex}
\usepackage[capposition=top,objectset=centering]{floatrow}
\usepackage{subcaption}
\usepackage{pdfpages}
\usepackage{pdflscape}
\usepackage{natbib}
\usepackage{bibunits}
\definecolor{maroon}{HTML}{990012}
\usepackage[bottom]{footmisc}
\usepackage[colorlinks=true,linkcolor=maroon,citecolor=maroon,urlcolor=maroon,anchorcolor=maroon]{hyperref}
\usepackage{marvosym}
\usepackage{makeidx}
\usepackage{tikz}
\usetikzlibrary{shapes}
\usepackage{setspace}
\usepackage{enumerate}
\usepackage{rotating}
\usepackage{tocloft}
\usepackage{epstopdf}
\usepackage[titletoc]{appendix}
\usepackage{framed}
\usepackage{comment}
\usepackage{xr}
\usepackage{titlesec}
\usepackage{footnote}
\usepackage{longtable}
\newlength{\tablewidth}
\setlength{\tablewidth}{9.3in}
\setcounter{secnumdepth}{4}
\usepackage{textgreek}

\titleformat{\paragraph}
{\normalfont\normalsize\bfseries}{\theparagraph}{1em}{}
\titlespacing*{\paragraph}
{0pt}{3.25ex plus 1ex minus .2ex}{1.5ex plus .2ex}
\makeatletter
\pretocmd\start@align
{%
  \let\everycr\CT@everycr
  \CT@start
}{}{}
\apptocmd{\endalign}{\CT@end}{}{}
\makeatother
%Watermark
\usepackage[printwatermark]{xwatermark}
\usepackage{lipsum}
\definecolor{lightgray}{RGB}{220,220,220}
%\newwatermark[allpages,color=lightgray,angle=45,scale=3,xpos=0,ypos=0]{Preliminary Draft}

%Further subsection level
\usepackage{titlesec}
\setcounter{secnumdepth}{4}
\titleformat{\paragraph}
{\normalfont\normalsize\bfseries}{\theparagraph}{1em}{}
\titlespacing*{\paragraph}
{0pt}{3.25ex plus 1ex minus .2ex}{1.5ex plus .2ex}

\setcounter{secnumdepth}{5}
\titleformat{\subparagraph}
{\normalfont\normalsize\bfseries}{\thesubparagraph}{1em}{}
\titlespacing*{\subparagraph}
{0pt}{3.25ex plus 1ex minus .2ex}{1.5ex plus .2ex}

%Functions
\DeclareMathOperator{\cov}{Cov}
\DeclareMathOperator{\corr}{Corr}
\DeclareMathOperator{\var}{Var}
\DeclareMathOperator{\plim}{plim}
\DeclareMathOperator*{\argmin}{arg\,min}
\DeclareMathOperator*{\argmax}{arg\,max}
\DeclareMathOperator{\supp}{supp}

%Math Environments
\newtheorem{theorem}{Theorem}
\newtheorem{claim}{Claim}
\newtheorem{condition}{Condition}
\renewcommand\thecondition{C--\arabic{condition}}
\newtheorem{algorithm}{Algorithm}
\newtheorem{assumption}{Assumption}
\newtheorem{remark}{Remark}
\renewcommand\theremark{R--\arabic{remark}}
\newtheorem{definition}[theorem]{Definition}
\newtheorem{hypothesis}[theorem]{Hypothesis}
\newtheorem{property}[theorem]{Property}
\newtheorem{example}[theorem]{Example}
\newtheorem{result}[theorem]{Result}
\newenvironment{proof}{\textbf{Proof:}}{$\bullet$}

%Commands
\newcommand\independent{\protect\mathpalette{\protect\independenT}{\perp}}
\def\independenT#1#2{\mathrel{\rlap{$#1#2$}\mkern2mu{#1#2}}}
\newcommand{\overbar}[1]{\mkern 1.5mu\overline{\mkern-1.5mu#1\mkern-1.5mu}\mkern 1.5mu}
\newcommand{\equald}{\ensuremath{\overset{d}{=}}}
\captionsetup[table]{skip=10pt}
%\makeindex

\setlength\parindent{20pt}
\setlength{\parskip}{0pt}

\newcolumntype{L}[1]{>{\raggedright\let\newline\\\arraybackslash\hspace{0pt}}m{#1}}
\newcolumntype{C}[1]{>{\centering\let\newline\\\arraybackslash\hspace{0pt}}m{#1}}
\newcolumntype{R}[1]{>{\raggedleft\let\newline\\\arraybackslash\hspace{0pt}}m{#1}}



%Logo
%\AddToShipoutPictureBG{%
%  \AtPageUpperLeft{\raisebox{-\height}{\includegraphics[width=1.5cm]{uchicago.png}}}
%}

\newcolumntype{L}[1]{>{\raggedright\let\newline\\\arraybackslash\hspace{0pt}}m{#1}}
\newcolumntype{C}[1]{>{\centering\let\newline\\\arraybackslash\hspace{0pt}}m{#1}}
\newcolumntype{R}[1]{>{\raggedleft\let\newline\\\arraybackslash\hspace{0pt}}m{#1}}

\newcommand{\mr}{\multirow}
\newcommand{\mc}{\multicolumn}

%\newcommand{\comment}[1]{}

\let\counterwithout\relax
\let\counterwithin\relax
\definecolor{maroon}{HTML}{4B0082}

\begin{document}
%\onehalfspacing

\noindent \textbf{Selection on Unobservables.}\\
\noindent Jorge Luis García \\
\noindent e-mail: jlgarcia@tamu.edu\\

\noindent The first framework in which selection on unobservables is acknowledge is the incidental truncation case where 
\begin{align}
	\textbf{(latent) selection variable} &: z_{i}^* \nonumber \\ 
	\textbf{(latent) dependent variable} &: y_{i}^* \nonumber
\end{align}

\noindent The selection variable determines whether $y_i$ is observed. These variables are modeled as
\begin{align}
	z_i^* & = \bm{w}_i \cdot \bm{\gamma} + u_i \nonumber \\
	y_i^* & = \bm{x}_i \cdot \bm{\beta} + e_i, 
\end{align}
\noindent where $\bm{w}_i := \left[ \bm{x}_i \ \vdots \  \bm{z}_i \right]$. $\bm{z}_i$ is a set of observed characteristics that appear in the selection latent, but not in the dependent-variable latent. The mapping between the latent is 
\begin{align}
z_i = \left\{
        \begin{array}{ll}
            1 & \quad z_i^* > 0 \\
            0 & \quad z_i^* < 0
        \end{array}
    \right.
\end{align}
\noindent and 
\begin{align}
y_i = \left\{
        \begin{array}{ll}
            y_i^* & \quad z_i = 1 \\
            \text{not observed} & \quad z_i = 0
        \end{array}
    \right.
\end{align}

\noindent The standard example is the following: $z_i^*$ is the propensity to join the labor force. If it is greater than a threshold, then $i$ works. Otherwise she does not. If she works, her wage ($y_i$) is observed.\\

\noindent The question is: Is it possible to recover consistent estimates of $\bm{\beta}$ by employing OLS on the observed sample? Exploring: 
\begin{align}
		\mathbb{E} \left[ \bm{\hat{\beta}}^{\text{OLS}} \mid \bm{w} \right]  & = \mathbb{E} \left[ { \left( \bm{x}' \bm{x} \right) }^{-1} \bm{x}' \bm{y} \mid \bm{w}, z_{i} = 1 \right] \nonumber \\
		    & = { \left( \bm{x}' \bm{x} \right) }^{-1} \bm{x}'	\mathbb{E} \left[ \bm{y} \mid \bm{w}, z_{i} = 1 \right]
\end{align}
\noindent Even if $\mathbb{E} \left[ e_i \mid \bm{w} \right] = 0$, OLS yields inconsistent estimates. That is because the distribution of $e_i$ could fundamentally differ between the subsample for which $z_i = 0$ and the subsample for which $z_i = 1$. That is, $\mathbb{E} \left[ e_i \mid \bm{w} \right] = 0$ does not imply that $\mathbb{E} \left[ e_i \mid \bm{w}, z_i = z \right] = 0$ for $z = 0,1$.\\ 

\noindent The objective is then to come up with a specification that enables obtaining consistent estimates of $\bm{\beta}$.\\ 

\noindent \textbf{Normality.} One alternative is to parameterize the joint distribution of $u_i$ and $e_i$ to obtain a ``control function.'' The argument is the following.\\

\noindent Let
\begin{align}
	\begin{pmatrix}
		e_i \\
		u_i
	\end{pmatrix} \sim \mathcal{N}
	\begin{bmatrix}
		\begin{pmatrix}
			0 \\ 
			0
		\end{pmatrix}, 
		\begin{pmatrix}
		\sigma_e^2 & \sigma_{ue} \\
		\sigma_u^2 & 1	
		\end{pmatrix}
	\end{bmatrix}. 
\end{align}

\noindent Then,
\begin{align}
	\mathbb{E} \left[ e_i \mid \bm{w}, z_i = 1 \right] & = \mathbb{E} \left[ e_i \mid \bm{w}, u_i > - \bm{w}_i \cdot \bm{\gamma} \right] \nonumber \\ 
	& = \frac{ \int \limits _{-\infty} ^{\infty} \int \limits _{-\bm{w}_i \cdot \bm{\gamma}} ^{\infty} e_i f \left( e_i, u_i \right) du_i de_i}{ \Phi \left( \bm{w}_i \cdot \bm{\gamma} \right)  } \nonumber \\
	& = \frac{ \int \limits _{-\infty} ^{\infty} \int \limits _{-\bm{w}_i \cdot \bm{\gamma}} ^{\infty} e_i f \left( e_i \mid  u_i \right) f \left( u_i \right)  du_i de_i}{ \Phi \left( \bm{w}_i \cdot \bm{\gamma} \right)  } \nonumber \\
	& = \frac{ \int \limits _{-\bm{w}_i \cdot \bm{\gamma}} ^{\infty} f \left( u_i \right)  \int \limits _{-\infty} ^{\infty} e_i f \left( e_i \mid  u_i \right) de_i du_i}{ \Phi \left( \bm{w}_i \cdot \bm{\gamma} \right)  } \nonumber \\
	& = \frac{ \sigma_{ue} \cdot \int \limits _{-\bm{w}_i \cdot \bm{\gamma}} ^{\infty} u_i f \left( u_i \right)  du_i}{ \Phi \left( \bm{w}_i \cdot \bm{\gamma} \right)  } \nonumber \\
	& = \sigma_{ue} \cdot \underbrace{ \frac{ \phi \left( \bm{w}_i \cdot \bm{\gamma} \right) } { \Phi \left( \bm{w}_i \cdot \bm{\gamma} \right) } }_{\text{Mill's ratio}}
\end{align}

\noindent This term is useful to ``correct'' the expectation of $y_i$, wich is biased by incidental truncation: 
\begin{align}
	\mathbb{E} \left[ y_i \mid \bm{w}, z_i = 1 \right] = \bm{x}_i \cdot \bm{\beta} + \mathbb{E} \left[ e_i \mid \bm{w}, z_i = 1 \right]
\end{align}

\noindent By including $\mathbb{E} \left[ e_i \mid \bm{w}, z_i = 1 \right]$ as a control, consistent estimates of $\beta$ would be obtained.\\

\noindent \textbf{Estimation.} Maximum likelihood enables jointly estimating the parameters characterizing the model for $y_i^*$ and the expectation correction or control function.\\

\noindent An intuitive two-step procedure is to estimate the parameters characterizing the selection model using a Probit, $\bm{\gamma}$, to then construct the Mill's ratio in a first step. In a second step, the parameters characterizing the dependent-variable, $\bm{\beta}$, adding the Mill's ratio ``as a control'' (from there the name control function). This not only yields a consistent estimate of $\bm{\beta}$ but also of $\sigma_{ue}$.\\

\noindent This intuitive approach actually does not require normality. If the correction procedure yields a consistent estimate of $\bm{\beta}$, then it is possible to use a non-parametric estimate of $\mathbb{E} \left[ e_i \mid \bm{w}, z_i = 1 \right]$ as a control function. Specification details are important: A non-parametric estimate could be, for example, a polynomial and could disable identifying the constant term in $\bm{\beta}$. 

\end{document}